\chapter{Department Policies}
\label{cha:department-policies}

\section{Advising}
\label{sec:advising}

\subsection{Pre-Professional Program}
\label{sec:pre-prof-progr}
We recognize the importance of providing a seamless and supportive academic pathway for students aspiring to excel in Electrical Engineering and Computer Engineering. The Pre-Professional Program in ECE has been meticulously designed to serve as a foundational bridge, allowing students to transition smoothly into their chosen major. Upon declaring their major in either Electrical Engineering or Computer Engineering, students gain access to the Pre-Professional Program, granting them the opportunity to enroll in 1000- and 2000-level courses within the ECE department.

The Pre-Professional Program in ECE aims to cultivate a strong educational foundation for future engineers. By enrolling in 1000- and 2000-level courses, students engage with fundamental concepts in electrical and computer engineering, laying the groundwork for advanced studies and specialization in their respective majors. This early exposure to essential topics such as circuits, digital systems, and programming languages equips students with the skills and knowledge necessary to tackle more complex challenges as they progress in their academic journey.

Moreover, being a part of the Pre-Professional Program provides students with a supportive community of peers who share similar interests and goals. Through collaborative projects, hands-on laboratories, and mentorship opportunities, students have the chance to enhance their problem-solving abilities and develop a keen understanding of real-world applications in electrical and computer engineering.

The Pre-Professional Program in ECE ensures that students receive personalized guidance from experienced faculty advisors who specialize in electrical and computer engineering. These advisors offer academic support, mentorship, and career guidance, empowering students to make informed decisions about their academic and professional trajectories. Additionally, the program facilitates interactions with industry professionals, research opportunities, and participation in engineering-related extracurricular activities, fostering a well-rounded educational experience.

\subsection{Professional Program}
\label{sec:professional-program}

The Professional Program  represents the advanced phase of academic studies for students majoring in Electrical Engineering and Computer Engineering . During their junior and senior years, students transition from general EE and CE coursework to more specialized courses. To gain access to these courses, students must be admitted into the Professional Program.

Upon completing the Pre-Professional Program, students are required to apply for acceptance into the Professional Program. Admission is competitive, and applicants must maintain a minimum Grade Point Average (GPA) of 2.8 in the Pre-Professional Required Courses. This eligibility criterion ensures a solid understanding of fundamental concepts and consistent academic performance.

Admittance into the Professional Program signifies a student's readiness for in-depth exploration of their chosen discipline. Under the guidance of faculty mentors, students engage in advanced coursework covering topics such as circuits, electronics, signal processing, programming languages, and hardware design. Practical experiences, including internships, projects, and research, enhance students' skills and prepare them for professional challenges in the electrical and computer engineering fields.

The Professional Program represents a commitment to academic excellence and rigorous training, providing students with the knowledge and practical expertise necessary for successful careers in Electrical and Computer Engineering.

The following is required to for admittance into the Professional Program:

\begin{itemize}
\item 2.8 GPA in the Pre-Professional Engineering courses
\item C or better in all Pre-Professional Engineering courses
\end{itemize}

\subsubsection{Required EE Pre-Professional Program courses}
\label{sec:required-ee-pre}



\begin{itemize}
    \item CHEM 1230 - Engineering Chemistry
    \item MATH 1210 - Calculus I
    \item MATH 1220 - Calculus II
    \item MATH 2210 - Calculus III
    \item ENGR 2240 - Dynamic Systems Engineering (or MATH 2250 or both MATH 2270 and MATH 2280)
    \item PHYS 2210 - Physics for Scientists and Engineers I
    \item PHYS 2220 - Physics for Scientists and Engineers II
    \item ECE 1000 - Introduction to Electrical Engineering (or ENGR 1000)
    \item ECE 1270 - Introduction to Electrical Circuits
    \item ECE 1400 - Fundamentals of Engineering Computing
    \item ECE 2260 - Fundamentals of Electrical Circuits
    \item ECE 2700 - Digital Circuits
\end{itemize}


\subsubsection{Required CE Pre-Professional Program courses}
\label{sec:required-ce-pre}

\begin{itemize}
    \item CS 1410 - Structured Computing in a Selected Language
    \item MATH 1210 - Calculus I
    \item MATH 1220 - Calculus II
    \item ENGR 2240 - Dynamic Systems Engineering (or MATH 2250 or both MATH 2270 and MATH 2280)
    \item PHYS 2210 - Physics for Scientists and Engineers I
    \item PHYS 2220 - Physics for Scientists and Engineers II
    \item ECE 1000 - Introduction to Electrical Engineering (or ENGR 1000)
    \item ECE 1270 - Introduction to Electrical Circuits
    \item ECE 1400 - Fundamentals of Engineering Computing
    \item ECE 2260 - Fundamentals of Electrical Circuits
    \item ECE 2700 - Digital Circuits
\end{itemize}

If you have completed the required courses, please submit the completed application \href{https://weber.edu/ece/advising.html}{form} to Dr. Eric Gibbons or Judy Smith.  

You may apply to the Professional Program prior to completing all of the required courses in the Pre-Professional Program.  This preliminary acceptance is contingent on the completion of the required courses with a C grade or better.  A preliminary acceptance to the Professional Program will allow you to enroll in upper-division courses.   If you have not yet completed a course in the pre-professional program list the grade as a C when filling out the application and include that placeholder grade when computing your GPA.  A GPA of 2.8 is still required for acceptance to the professional program.  If your GPA falls below 2.8 with these placeholder C grades, wait until you have the final grades after taking the courses before you apply.  

If you are a transfer student, please reach out to the program coordinator or advisors on how to fill out the application.  

\subsection{Dismissal from the program}
\label{sec:dism-from-progr}

At Weber State University, we are deeply committed to nurturing the academic growth and success of our students within the ECE Program. To maintain the rigorous standards and high-quality education that our program offers, there is a specific policy concerning the repetition of courses.

If a student fails or withdraws from the same required ECE course twice during their enrollment in the program, they will be dimissed from either the ECE Pre-Preprofessional or Professional Program. This policy is implemented to encourage students to actively engage in their studies and to seek appropriate assistance when encountering challenges.

\subsubsection{Rationale for the dimissal policy}
\label{sec:rati-dimiss-policy}


The rationale behind this policy is multifaceted. Firstly, it ensures that students are making consistent academic progress, a crucial requirement for successful completion of any program. By upholding this standard, we provide a clear academic pathway to graduation, ensuring that our graduates from Weber State University are well-prepared and knowledgeable as they embark on a career in industry.

Secondly, the policy encourages students to utilize the comprehensive support services available at Weber State University. We believe in offering a robust support system, including academic advising, tutoring, and counseling, to help students overcome obstacles and thrive in their studies. Seeking assistance promptly often enables students to address their concerns effectively, preventing the need for course repetition and allowing them to maintain their progress in the ECE Program.

In situations where a student faces exceptional circumstances that may have affected their academic performance, there is an established appeals process. Students can provide relevant documentation and appeal to the Academic Review Board, ensuring that individual cases are thoroughly examined and decisions are made fairly and equitably.

By adhering to this policy, we uphold the reputation and integrity of Weber State University's ECE Program. This commitment ensures that our graduates not only possess exceptional technical skills but also demonstrate resilience, determination, and adaptability, making them well-prepared for the challenges of the professional world in Electrical and Computer Engineering.


\section{Academic integrity}
\label{sec:academic-integrity}

In the ECE Department, we uphold the highest standards of academic integrity and ethical conduct. As future engineers and innovators, it is paramount that our students understand the significance of maintaining honesty, trust, and respect within the academic community. Academic integrity violations undermine the core values of our department and erode the foundation upon which our educational system is built.

One of the specific areas of concern in our department revolves around the unauthorized sharing of course materials, including source code, assignments, and project files. While collaboration and knowledge sharing are encouraged within the boundaries of each course, it is imperative that students recognize the importance of respecting the intellectual property rights associated with their coursework. This includes refraining from posting course code or any other class-related materials publicly, even after the course has concluded.

\subsection{Consequences}
\label{sec:consequences}

Academic integrity violations are treated with the utmost seriousness in the ECE Department. Such actions compromise the learning process, diminish the value of the education received, and can lead to severe consequences. Students found in violation of these policies may face disciplinary actions ranging from receiving a failing grade for the assignment or course to suspension or expulsion from the university. Additionally, academic integrity violations tarnish a student's reputation, potentially impacting their future career prospects and professional relationships.

\subsection{Promoting a culture of integrity and responsiblity}
\label{sec:prom-cult-integr}

To promote a culture of academic integrity and responsibility, the ECE Department emphasizes the importance of understanding and abiding by university policies related to plagiarism, cheating, and unauthorized distribution of course materials. Faculty members employ various tools and techniques, including plagiarism detection software, to identify potential violations and ensure a fair and just academic environment for all students.

Furthermore, students are encouraged to actively engage with the principles of academic integrity and to seek guidance from faculty members and academic advisors if they are uncertain about the appropriate boundaries of collaboration and knowledge sharing. By fostering an environment of open communication, mutual respect, and ethical conduct, we can collectively uphold the integrity of our department and nurture a community of engineers who are not only technically proficient but also ethically responsible leaders in their field.


\subsection{Examples of academic misbehavior}
\label{sec:exampl-acad-misb}

The following is an incomplete list of examples of academic misbehavior.  Please note that this list is not exhaustive, and there are other forms of academic integrity violations that may occur. It serves as a handful of examples to highlight the various types of violations that students should be aware of.

\begin{itemize}
    \item Plagiarism: Presenting someone else's work, ideas, or intellectual property as one's own without proper citation.
    \item Cheating on Exams: Using unauthorized materials, devices, or communication during exams to gain an unfair advantage.
    \item Unauthorized Collaboration: Working together on assignments, projects, or exams without explicit permission from the instructor.
    \item Fabrication: Creating false data, citations, or information to support one's academic work.
    \item Submitting Work from Previous Semesters: Submitting work that was previously graded for another course without the current instructor's consent.
    \item Using Someone Else's Code or Design Without Attribution: Utilizing programming code, circuit designs, or other intellectual property created by others without proper acknowledgment.
    \item Breach of Exam Security: Sharing exam questions, answers, or other sensitive information with students who have not yet taken the exam.
    \item Forging Signatures or Approval: Falsifying signatures, approval, or any official documents related to academic processes.
    \item Unauthorized Access to Exam or Grading Systems: Gaining unauthorized access to exam papers, grading systems, or any other secure academic information.
    \item Publicly Posting Course Materials: Posting old course materials, solutions, or code, even after the course is complete.
\end{itemize}

%%% Local Variables:
%%% mode: latex
%%% TeX-master: t
%%% End:
