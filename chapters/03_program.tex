
\chapter{Program overview}
\label{cha:program-overview}

\section{Mission and goals}
\label{sec:mission-goals}

The mission of the Electrical and Computer Engineering (ECE) program at Weber State is to educate students to become skilled and innovative engineers who contribute effectively to society. Our program aims to provide a comprehensive understanding of electrical and computer engineering principles, foster critical thinking and problem-solving skills, and prepare graduates for successful careers in various industries or advanced studies in graduate school.

\subsection{Program learning outcomes}
\label{sec:progr-learn-outc}


Upon completion of the ECE program, students are expected to demonstrate the following learning outcomes:

\begin{itemize}
\item Apply mathematical, scientific, and engineering principles to solve complex electrical and computer engineering problems.
\item Design and conduct experiments, as well as analyze and interpret data in electrical and computer engineering domains.
\item Design systems and components, and integrate them into complex, realistic contexts.
\item Work effectively in multidisciplinary teams, demonstrating communication, leadership, and project management skills.
\item Recognize the ethical and social implications of engineering solutions and make informed decisions in professional practice.
\item Engage in lifelong learning through continuous professional development and advanced studies.
\end{itemize}

\section{Degree programs}
\label{sec:degree-programs}

The Department of Electrical and Computer Engineering offers the following undergraduate degrees:

\begin{itemize}
\item Bachelor of Science in Electrical Engineering (BSEE)
\item Bachelor of Science in Computer Engineering (BSCE)
\item Bachelor of Science in Biomedical Engineering (BSBME)
\end{itemize}


\subsection{Electrical Engineering}
\label{sec:electrical-engineering}

The following  is a suggested plan to complete the the BSEE degree in four years. Meet with an academic advisor to create a specific plan that best fits your academic needs. Remember, taking an average of 15 credit hours per semester facilitates timely graduation.


{\footnotesize
  \def\arraystretch{1.1}
  \begin{longtable}{| l | c | l | l |}
    \rowcolor{Purple}
    \multicolumn{1}{c}{{\color{white}\textbf{Course}}} &
    \multicolumn{1}{c}{{\color{white}\textbf{Cr.}}} &
    \multicolumn{1}{c}{{\color{white}\textbf{Prerequisites}}} &
    \multicolumn{1}{c}{{\color{white}\textbf{Offered}}} \\
    \csname @@input\endcsname assets/ee_grad_table.tex
  \end{longtable}
}

$^{*}$ The following alternatives are acceptable:
\begin{itemize}
\item ECE 3510 Power Systems may be taken in lieu of ECE 3610.
\item ECE 5210 Digital Signal Processing may be taken in lieu of ECE 4100, though ECE 5210 is typically taught during the spring semester.
\end{itemize}

Acceptable Senior Electives are given below.  Senior Electives are usually taught on an every-other-year basis.

{\footnotesize
\def\arraystretch{1.1}
\begin{longtable}{| l | c | l |}
  \rowcolor{Purple}
  \multicolumn{1}{c}{{\color{white}\textbf{Course}}} &
  \multicolumn{1}{c}{{\color{white}\textbf{Cr.}}} &
  \multicolumn{1}{c}{{\color{white}\textbf{Prerequisites}}} \\
  \csname @@input\endcsname assets/ee_electives_table.tex
  
\end{longtable}
}

Note:  at most one 3000-level course can count as an acceptable Senior Level elective.

\subsection{Computer Engineering}
\label{sec:computer-engineering}

The following  is a suggested plan to complete the the BSCE degree in four years. Meet with an academic advisor to create a specific plan that best fits your academic needs. Remember, taking an average of 15 credit hours per semester facilitates timely graduation.


{\footnotesize
  \def\arraystretch{1.1}
  \begin{longtable}{| l | c | l | l |}
    \rowcolor{Purple}
    \multicolumn{1}{c}{{\color{white}\textbf{Course}}} &
    \multicolumn{1}{c}{{\color{white}\textbf{Cr.}}} &
    \multicolumn{1}{c}{{\color{white}\textbf{Prerequisites}}} &
    \multicolumn{1}{c}{{\color{white}\textbf{Offered}}} \\
    \csname @@input\endcsname assets/ce_grad_table.tex
  \end{longtable}
}

Acceptable Senior Electives are given below.  Senior Electives are usually taught on an every-other-year basis.

{\footnotesize
\def\arraystretch{1.1}
\begin{longtable}{| l | c | l |}
  \rowcolor{Purple}
  \multicolumn{1}{c}{{\color{white}\textbf{Course}}} &
  \multicolumn{1}{c}{{\color{white}\textbf{Cr.}}} &
  \multicolumn{1}{c}{{\color{white}\textbf{Prerequisites}}} \\
  \csname @@input\endcsname assets/ce_electives_table.tex
  
\end{longtable}
}

Note:  at most one 3000-level course can count as an acceptable Senior Level elective.




%%% Local Variables:
%%% mode: latex
%%% TeX-master: "../main"
%%% End:
