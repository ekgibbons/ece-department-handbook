\documentclass[12pt]{report}

%%%%%%%%%%%%
% Packages %
%%%%%%%%%%%%

\usepackage[english]{babel}
\usepackage{packages/sleek}
\usepackage{packages/sleek-title}
\usepackage{packages/sleek-theorems}
% \usepackage{packages/sleek-listings}
\usepackage{packages/grad_table}
\usepackage{listings}
\usepackage{dirtree}
\usepackage{longtable}
\usepackage{caption}
\usepackage{subcaption}


\usepackage{tikz}
\usepackage{circuitikz}
\usepackage{pgfplots}

\usetikzlibrary{shapes.geometric,fit}

\newcommand{\suma}{\Large$+$}
\newcommand{\inte}{$\displaystyle \int$}
\newcommand{\derv}{\huge$\frac{d}{dt}$}


%%%%%%%%%%%%%%
% Title-page %
%%%%%%%%%%%%%%

\logo{./resources/pdf/ece_horiz.pdf}
\institute{Weber State University}
\faculty{Electrical and Computer Engineering}
%\department{Department of Anything but Psychology}
\title{Department Handbook}
\subtitle{2023-2024}
\author{\textit{Author}\\Eric \textsc{Gibbons}}
\supervisor{Version 0.0.0}
%\context{Well, I was bored...}
\date{\today}

%%%%%%%%%%%%%%%%
% Bibliography %
%%%%%%%%%%%%%%%%


\addbibresource{./resources/bib/references.bib}
% \bibliographystyle{plain}
% \bibliography{./resources/bib/references.bib}

%%%%%%%%%%
% Others %
%%%%%%%%%%

\definecolor{codegreen}{rgb}{0,0.6,0}
\definecolor{codegray}{rgb}{0.5,0.5,0.5}
\definecolor{codepurple}{rgb}{0.58,0,0.82}
\definecolor{backcolour}{rgb}{0.95,0.95,0.92}

\lstdefinestyle{mystyle}{
  backgroundcolor=\color{backcolour},   
  commentstyle=\color{codegreen},
  keywordstyle=\color{magenta},
  numberstyle=\tiny\color{codegray},
  stringstyle=\color{codepurple},
  basicstyle=\ttfamily\footnotesize,
  breakatwhitespace=false,         
  breaklines=true,                 
  captionpos=b,                    
  keepspaces=true,                 
  % numbers=left,                    
  numbersep=5pt,                  
  showspaces=false,                
  showstringspaces=false,
  showtabs=false,                  
  tabsize=2,
}

\lstset{style=mystyle}

\renewcommand{\abstractname}{Acknowledgements}
\setcounter{chapter}{0}

%%%%%%%%%%%%
% Document %
%%%%%%%%%%%%

\begin{document}
    \maketitle
    \tableofcontents

    % \renewcommand{\abstractname}{Acknowledgements}


\begin{abstract}
I am grateful to Jeff Ward, Fon Brown, and Chris Trampel for sharing their lab exercises. These materials were used to create much of the current lab sequence for this course.
\end{abstract}

%%% Local Variables:
%%% mode: latex
%%% TeX-master: "../main"
%%% End:

    \chapter{Introduction}
\label{cha:introduction}

\section{Chair's message}
\label{sec:chairs-message}

Welcome to the Department of Electrical and Computer Engineering at Weber State University! As a student in our Electrical and Computer Engineering (ECE) program, you are embarking on an exciting journey of discovery, innovation, and personal growth. This handbook is designed to provide you with essential information about the program’s curriculum, academic policies, resources, and opportunities available to you during your undergraduate studies. We encourage you to familiarize yourself with the contents of this handbook and refer to it whenever you have questions about your academic journey.

\section{About ECE}
\label{sec:about-departm-electr}

The Department of Electrical and Computer Engineering at Weber State University is committed to providing high-quality education and fostering a supportive learning environment for aspiring electrical and computer engineers. Our distinguished faculty members are experts in various fields of electrical and computer engineering, and they are dedicated to helping you succeed academically and professionally. Our state-of-the-art laboratories, cutting-edge research initiatives, and collaborative learning opportunities ensure that you receive a well-rounded education and are prepared to tackle the challenges of the modern world.

In this handbook, you will find detailed information about the ECE program's curriculum, academic policies, available resources, and the various opportunities that will enhance your learning experience. We encourage you to take full advantage of the resources and support services offered by the department to make the most of your time here.

We wish you a fulfilling and successful academic journey in the Department of Electrical and Computer Engineering at Weber State.

%%% local Variables:
%%% mode: latex
%%% TeX-master: "../main"
%%% End:

    
\chapter{Department Organization}
\label{cha:departm-organ}

\section{Department structure}
\label{sec:department-structure}

The Electrical and Computer Engineering Department is structured to provide students with a robust academic foundation and real-world applications. The department is led by a dedicated team of experienced faculty members and administrators who are passionate about advancing the field of electrical and computer engineering.
\begin{flushleft}
\begin{tabular}{p{8cm}p{6cm}}
\textbf{Position} & \textbf{Name} \\
Chair & Fon Brown, PhD \\
Undergraduate Program Coordinator & Eric Gibbons, PhD \\
Graduate Program Director & Jutin Jackson, PhD \\
\end{tabular}
\end{flushleft}

The ECE faculty members are experts in various fields, including electronics, communications, computer systems, robotics, and power systems. Students are encouraged to interact with faculty members, attend their office hours, and seek mentorship for academic and research pursuits.

\subsection{ECE Advising and Support}
\label{sec:ece-faculty-direct}


In addition to faculty members, the department has dedicated staff members who assist with administrative, advising, and technical support services. The staff directory provides contact information for administrative assistants, advisors, laboratory technicians, and other support personnel.

\begin{flushleft}
\begin{tabular}{p{8cm}p{6cm}}
\textbf{Position} & \textbf{Name} \\
Administrative assistant & Judy Smith \\
Academic advisor & Aimee Smith \\
Graduate enrollment director & Rainie Ingram \\
\end{tabular}
\end{flushleft}


\subsection{Office hours and contact information}
\label{sec:office-hours-contact}

Faculty and staff members have designated office hours during which students can ask questions, seek clarification, and discuss academic matters. Office hours are posted on the department's website and outside individual faculty offices. Contact information for faculty and staff members is also available online.


\section{Department facilities}
\label{sec:departm-facil}

Our department boasts state-of-the-art laboratories equipped with the latest technology.


providing students with hands-on experience in areas such as:

Digital Signal Processing
Robotics and Automation
Integrated Circuit Design
Power Systems
Wireless Communication


\section{Student resources}
\label{sec:student-resources}

\subsection{Academic advising}
\label{sec:academic-advising}


Our dedicated academic advisors assist students in course selection, career planning, and graduate school applications. Regular advising sessions ensure that students stay on track to meet their academic and career goals.

\subsection{Career services}
\label{sec:career-services}

The ECE Department collaborates with the college's Career Services center to provide internship and job placement opportunities, career fairs, and networking events with industry professionals. Workshops on resume building and interview skills are also offered.

\subsection{Student clubs and organizations}
\label{sec:stud-clubs-organ}

Engage with like-minded peers and enhance your leadership skills by joining student organizations such as the IEEE Student Branch and Tau Beta Pi. These organizations offer a platform to collaborate on projects, attend conferences, and participate in competitions, enriching your overall university experience.

%%% Local Variables:
%%% mode: latex
%%% TeX-master: "../main"
%%% End:

    
\chapter{Program overview}
\label{cha:program-overview}

\section{Mission and goals}
\label{sec:mission-goals}

The mission of the Electrical and Computer Engineering (ECE) program at Weber State is to educate students to become skilled and innovative engineers who contribute effectively to society. Our program aims to provide a comprehensive understanding of electrical and computer engineering principles, foster critical thinking and problem-solving skills, and prepare graduates for successful careers in various industries or advanced studies in graduate school.

\subsection{Program learning outcomes}
\label{sec:progr-learn-outc}


Upon completion of the ECE program, students are expected to demonstrate the following learning outcomes:

\begin{itemize}
\item Apply mathematical, scientific, and engineering principles to solve complex electrical and computer engineering problems.
\item Design and conduct experiments, as well as analyze and interpret data in electrical and computer engineering domains.
\item Design systems and components, and integrate them into complex, realistic contexts.
\item Work effectively in multidisciplinary teams, demonstrating communication, leadership, and project management skills.
\item Recognize the ethical and social implications of engineering solutions and make informed decisions in professional practice.
\item Engage in lifelong learning through continuous professional development and advanced studies.
\end{itemize}

\section{Degree programs}
\label{sec:degree-programs}

The Department of Electrical and Computer Engineering offers the following undergraduate degrees:

\begin{itemize}
\item Bachelor of Science in Electrical Engineering (BSEE)
\item Bachelor of Science in Computer Engineering (BSCE)
\item Bachelor of Science in Biomedical Engineering (BSBME)
\end{itemize}


\subsection{Electrical Engineering}
\label{sec:electrical-engineering}

The following  is a suggested plan to complete the the BSEE degree in four years. Meet with an academic advisor to create a specific plan that best fits your academic needs. Remember, taking an average of 15 credit hours per semester facilitates timely graduation.


{\footnotesize
  \def\arraystretch{1.1}
  \begin{longtable}{| l | c | l | l |}
    \rowcolor{Purple}
    \multicolumn{1}{c}{{\color{white}\textbf{Course}}} &
    \multicolumn{1}{c}{{\color{white}\textbf{Cr.}}} &
    \multicolumn{1}{c}{{\color{white}\textbf{Prerequisites}}} &
    \multicolumn{1}{c}{{\color{white}\textbf{Offered}}} \\
    \csname @@input\endcsname assets/ee_grad_table.tex
  \end{longtable}
}

$^{*}$ The following alternatives are acceptable:
\begin{itemize}
\item ECE 3510 Power Systems may be taken in lieu of ECE 3610.
\item ECE 5210 Digital Signal Processing may be taken in lieu of ECE 4100, though ECE 5210 is typically taught during the spring semester.
\end{itemize}

Acceptable Senior Electives are given below.  Senior Electives are usually taught on an every-other-year basis.

{\footnotesize
\def\arraystretch{1.1}
\begin{longtable}{| l | c | l |}
  \rowcolor{Purple}
  \multicolumn{1}{c}{{\color{white}\textbf{Course}}} &
  \multicolumn{1}{c}{{\color{white}\textbf{Cr.}}} &
  \multicolumn{1}{c}{{\color{white}\textbf{Prerequisites}}} \\
  \csname @@input\endcsname assets/ee_electives_table.tex
  
\end{longtable}
}

Note:  at most one 3000-level course can count as an acceptable Senior Level elective.

\subsection{Computer Engineering}
\label{sec:computer-engineering}

The following  is a suggested plan to complete the the BSCE degree in four years. Meet with an academic advisor to create a specific plan that best fits your academic needs. Remember, taking an average of 15 credit hours per semester facilitates timely graduation.


{\footnotesize
  \def\arraystretch{1.1}
  \begin{longtable}{| l | c | l | l |}
    \rowcolor{Purple}
    \multicolumn{1}{c}{{\color{white}\textbf{Course}}} &
    \multicolumn{1}{c}{{\color{white}\textbf{Cr.}}} &
    \multicolumn{1}{c}{{\color{white}\textbf{Prerequisites}}} &
    \multicolumn{1}{c}{{\color{white}\textbf{Offered}}} \\
    \csname @@input\endcsname assets/ce_grad_table.tex
  \end{longtable}
}

Acceptable Senior Electives are given below.  Senior Electives are usually taught on an every-other-year basis.

{\footnotesize
\def\arraystretch{1.1}
\begin{longtable}{| l | c | l |}
  \rowcolor{Purple}
  \multicolumn{1}{c}{{\color{white}\textbf{Course}}} &
  \multicolumn{1}{c}{{\color{white}\textbf{Cr.}}} &
  \multicolumn{1}{c}{{\color{white}\textbf{Prerequisites}}} \\
  \csname @@input\endcsname assets/ce_electives_table.tex
  
\end{longtable}
}

Note:  at most one 3000-level course can count as an acceptable Senior Level elective.




%%% Local Variables:
%%% mode: latex
%%% TeX-master: "../main"
%%% End:

    \chapter{Department Policies}
\label{cha:department-policies}

\section{Advising}
\label{sec:advising}

\subsection{Pre-Professional Program}
\label{sec:pre-prof-progr}
We recognize the importance of providing a seamless and supportive academic pathway for students aspiring to excel in Electrical Engineering and Computer Engineering. The Pre-Professional Program in ECE has been meticulously designed to serve as a foundational bridge, allowing students to transition smoothly into their chosen major. Upon declaring their major in either Electrical Engineering or Computer Engineering, students gain access to the Pre-Professional Program, granting them the opportunity to enroll in 1000- and 2000-level courses within the ECE department.

The Pre-Professional Program in ECE aims to cultivate a strong educational foundation for future engineers. By enrolling in 1000- and 2000-level courses, students engage with fundamental concepts in electrical and computer engineering, laying the groundwork for advanced studies and specialization in their respective majors. This early exposure to essential topics such as circuits, digital systems, and programming languages equips students with the skills and knowledge necessary to tackle more complex challenges as they progress in their academic journey.

Moreover, being a part of the Pre-Professional Program provides students with a supportive community of peers who share similar interests and goals. Through collaborative projects, hands-on laboratories, and mentorship opportunities, students have the chance to enhance their problem-solving abilities and develop a keen understanding of real-world applications in electrical and computer engineering.

The Pre-Professional Program in ECE ensures that students receive personalized guidance from experienced faculty advisors who specialize in electrical and computer engineering. These advisors offer academic support, mentorship, and career guidance, empowering students to make informed decisions about their academic and professional trajectories. Additionally, the program facilitates interactions with industry professionals, research opportunities, and participation in engineering-related extracurricular activities, fostering a well-rounded educational experience.

\subsection{Professional Program}
\label{sec:professional-program}

The Professional Program  represents the advanced phase of academic studies for students majoring in Electrical Engineering and Computer Engineering . During their junior and senior years, students transition from general EE and CE coursework to more specialized courses. To gain access to these courses, students must be admitted into the Professional Program.

Upon completing the Pre-Professional Program, students are required to apply for acceptance into the Professional Program. Admission is competitive, and applicants must maintain a minimum Grade Point Average (GPA) of 2.8 in the Pre-Professional Required Courses. This eligibility criterion ensures a solid understanding of fundamental concepts and consistent academic performance.

Admittance into the Professional Program signifies a student's readiness for in-depth exploration of their chosen discipline. Under the guidance of faculty mentors, students engage in advanced coursework covering topics such as circuits, electronics, signal processing, programming languages, and hardware design. Practical experiences, including internships, projects, and research, enhance students' skills and prepare them for professional challenges in the electrical and computer engineering fields.

The Professional Program represents a commitment to academic excellence and rigorous training, providing students with the knowledge and practical expertise necessary for successful careers in Electrical and Computer Engineering.

The following is required to for admittance into the Professional Program:

\begin{itemize}
\item 2.8 GPA in the Pre-Professional Engineering courses
\item C or better in all Pre-Professional Engineering courses
\end{itemize}

\subsubsection{Required EE Pre-Professional Program courses}
\label{sec:required-ee-pre}



\begin{itemize}
    \item CHEM 1230 - Engineering Chemistry
    \item MATH 1210 - Calculus I
    \item MATH 1220 - Calculus II
    \item MATH 2210 - Calculus III
    \item ENGR 2240 - Dynamic Systems Engineering (or MATH 2250 or both MATH 2270 and MATH 2280)
    \item PHYS 2210 - Physics for Scientists and Engineers I
    \item PHYS 2220 - Physics for Scientists and Engineers II
    \item ECE 1000 - Introduction to Electrical Engineering (or ENGR 1000)
    \item ECE 1270 - Introduction to Electrical Circuits
    \item ECE 1400 - Fundamentals of Engineering Computing
    \item ECE 2260 - Fundamentals of Electrical Circuits
    \item ECE 2700 - Digital Circuits
\end{itemize}


\subsubsection{Required CE Pre-Professional Program courses}
\label{sec:required-ce-pre}

\begin{itemize}
    \item CS 1410 - Structured Computing in a Selected Language
    \item MATH 1210 - Calculus I
    \item MATH 1220 - Calculus II
    \item ENGR 2240 - Dynamic Systems Engineering (or MATH 2250 or both MATH 2270 and MATH 2280)
    \item PHYS 2210 - Physics for Scientists and Engineers I
    \item PHYS 2220 - Physics for Scientists and Engineers II
    \item ECE 1000 - Introduction to Electrical Engineering (or ENGR 1000)
    \item ECE 1270 - Introduction to Electrical Circuits
    \item ECE 1400 - Fundamentals of Engineering Computing
    \item ECE 2260 - Fundamentals of Electrical Circuits
    \item ECE 2700 - Digital Circuits
\end{itemize}

If you have completed the required courses, please submit the completed application \href{https://weber.edu/ece/advising.html}{form} to Dr. Eric Gibbons or Judy Smith.  

You may apply to the Professional Program prior to completing all of the required courses in the Pre-Professional Program.  This preliminary acceptance is contingent on the completion of the required courses with a C grade or better.  A preliminary acceptance to the Professional Program will allow you to enroll in upper-division courses.   If you have not yet completed a course in the pre-professional program list the grade as a C when filling out the application and include that placeholder grade when computing your GPA.  A GPA of 2.8 is still required for acceptance to the professional program.  If your GPA falls below 2.8 with these placeholder C grades, wait until you have the final grades after taking the courses before you apply.  

If you are a transfer student, please reach out to the program coordinator or advisors on how to fill out the application.  

\subsection{Dismissal from the program}
\label{sec:dism-from-progr}

At Weber State University, we are deeply committed to nurturing the academic growth and success of our students within the ECE Program. To maintain the rigorous standards and high-quality education that our program offers, there is a specific policy concerning the repetition of courses.

If a student fails or withdraws from the same required ECE course twice during their enrollment in the program, they will be dimissed from either the ECE Pre-Preprofessional or Professional Program. This policy is implemented to encourage students to actively engage in their studies and to seek appropriate assistance when encountering challenges.

\subsubsection{Rationale for the dimissal policy}
\label{sec:rati-dimiss-policy}


The rationale behind this policy is multifaceted. Firstly, it ensures that students are making consistent academic progress, a crucial requirement for successful completion of any program. By upholding this standard, we provide a clear academic pathway to graduation, ensuring that our graduates from Weber State University are well-prepared and knowledgeable as they embark on a career in industry.

Secondly, the policy encourages students to utilize the comprehensive support services available at Weber State University. We believe in offering a robust support system, including academic advising, tutoring, and counseling, to help students overcome obstacles and thrive in their studies. Seeking assistance promptly often enables students to address their concerns effectively, preventing the need for course repetition and allowing them to maintain their progress in the ECE Program.

In situations where a student faces exceptional circumstances that may have affected their academic performance, there is an established appeals process. Students can provide relevant documentation and appeal to the Academic Review Board, ensuring that individual cases are thoroughly examined and decisions are made fairly and equitably.

By adhering to this policy, we uphold the reputation and integrity of Weber State University's ECE Program. This commitment ensures that our graduates not only possess exceptional technical skills but also demonstrate resilience, determination, and adaptability, making them well-prepared for the challenges of the professional world in Electrical and Computer Engineering.


\section{Academic integrity}
\label{sec:academic-integrity}

In the ECE Department, we uphold the highest standards of academic integrity and ethical conduct. As future engineers and innovators, it is paramount that our students understand the significance of maintaining honesty, trust, and respect within the academic community. Academic integrity violations undermine the core values of our department and erode the foundation upon which our educational system is built.

One of the specific areas of concern in our department revolves around the unauthorized sharing of course materials, including source code, assignments, and project files. While collaboration and knowledge sharing are encouraged within the boundaries of each course, it is imperative that students recognize the importance of respecting the intellectual property rights associated with their coursework. This includes refraining from posting course code or any other class-related materials publicly, even after the course has concluded.

\subsection{Consequences}
\label{sec:consequences}

Academic integrity violations are treated with the utmost seriousness in the ECE Department. Such actions compromise the learning process, diminish the value of the education received, and can lead to severe consequences. Students found in violation of these policies may face disciplinary actions ranging from receiving a failing grade for the assignment or course to suspension or expulsion from the university. Additionally, academic integrity violations tarnish a student's reputation, potentially impacting their future career prospects and professional relationships.

\subsection{Promoting a culture of integrity and responsiblity}
\label{sec:prom-cult-integr}

To promote a culture of academic integrity and responsibility, the ECE Department emphasizes the importance of understanding and abiding by university policies related to plagiarism, cheating, and unauthorized distribution of course materials. Faculty members employ various tools and techniques, including plagiarism detection software, to identify potential violations and ensure a fair and just academic environment for all students.

Furthermore, students are encouraged to actively engage with the principles of academic integrity and to seek guidance from faculty members and academic advisors if they are uncertain about the appropriate boundaries of collaboration and knowledge sharing. By fostering an environment of open communication, mutual respect, and ethical conduct, we can collectively uphold the integrity of our department and nurture a community of engineers who are not only technically proficient but also ethically responsible leaders in their field.


\subsection{Examples of academic misbehavior}
\label{sec:exampl-acad-misb}

The following is an incomplete list of examples of academic misbehavior.  Please note that this list is not exhaustive, and there are other forms of academic integrity violations that may occur. It serves as a handful of examples to highlight the various types of violations that students should be aware of.

\begin{itemize}
    \item Plagiarism: Presenting someone else's work, ideas, or intellectual property as one's own without proper citation.
    \item Cheating on Exams: Using unauthorized materials, devices, or communication during exams to gain an unfair advantage.
    \item Unauthorized Collaboration: Working together on assignments, projects, or exams without explicit permission from the instructor.
    \item Fabrication: Creating false data, citations, or information to support one's academic work.
    \item Submitting Work from Previous Semesters: Submitting work that was previously graded for another course without the current instructor's consent.
    \item Using Someone Else's Code or Design Without Attribution: Utilizing programming code, circuit designs, or other intellectual property created by others without proper acknowledgment.
    \item Breach of Exam Security: Sharing exam questions, answers, or other sensitive information with students who have not yet taken the exam.
    \item Forging Signatures or Approval: Falsifying signatures, approval, or any official documents related to academic processes.
    \item Unauthorized Access to Exam or Grading Systems: Gaining unauthorized access to exam papers, grading systems, or any other secure academic information.
    \item Publicly Posting Course Materials: Posting old course materials, solutions, or code, even after the course is complete.
\end{itemize}

%%% Local Variables:
%%% mode: latex
%%% TeX-master: t
%%% End:

    % \input{chapters/02_intro_to_pybind}
    % \input{chapters/03_impulse_response}
    % \input{chapters/04_convolution}
    % \input{chapters/05_fourier_series}
    % \input{chapters/06_periodic_response}
    % \input{chapters/07_aliasing}
    % \input{chapters/08_fft}
    % \input{chapters/09_freq_response}
    % \input{chapters/10_lpf}
    % \input{chapters/11_discrete_systems}
    
    % \printbibliography
%    \bibliographystyle{plain} % We choose the "plain" reference style

    
    % \chapter{Introduction}

    % Sleek Template is a minimal collection of \LaTeX{} packages and settings that ease the writing of beautiful documents. While originally meant for theses, it is perfectly suitable for project reports, articles, syntheses, etc. -- with a few adjustments, like margins.

    % It is composed of four separate packages -- \texttt{sleek}, \texttt{sleek-title}, \texttt{sleek-theorems} and \texttt{sleek-listings} -- each of which can be used individually.

    % \begin{lstlisting}[style=latexFrameTB, caption={Example of Sleek Template packages usage.}, gobble=8]
    %     \usepackage[english]{babel}
    %     \usepackage[noheader]{packages/sleek}
    %     \usepackage{packages/sleek-title}
    % \end{lstlisting}

    % \blindfootnote{If you are a \LaTeX{} beginner consider the excellent \href{https://www.overleaf.com/learn}{Overleaf tutorial}. Also, there are a lot of symbols available in \LaTeX{} and, therefore, in this template. I recommend the use of \enquote{The Comprehensive \LaTeX{} Symbol List} \cite{pakin2020comprehensive} for searching symbols.}

    % \chapter{Features}

    % \section{\texttt{sleek}}

    % \texttt{sleek} is the main package. It imports the packages (\emph{cf.} Table \ref{tab:sleek_relevant_packages}) and setups the settings that make Sleek Template easy to use.

    % There are three available options to the \texttt{sleek} package :

    % \begin{enumerate}[noitemsep]
    %     \item \texttt{parindent} add indentation to the first line of paragraphs;
    %     \item \texttt{noheader} removes the document header;
    %     \item \texttt{french} changes the decimal sign to a comma and translates some captions.
    % \end{enumerate}

    % But nothing prevents you to tweak the settings to your liking in the source code.

    % \subsection{Mathematics}

    % This template uses \texttt{amsmath} and \texttt{amssymb}, which are the de-facto standard for typesetting mathematics. Additionally, \texttt{esint} provides alternative integral symbols (\emph{cf.} Table 78 in \cite{pakin2020comprehensive}) and \texttt{bm} is used for bold math symbols like vectors (see \eqref{eq:gauss_law}).

    % A few custom macros have also been added such as \texttt{\tbs{}rbk}, \texttt{\tbs{}sbk} and \texttt{\tbs{}cbk} for respectively round, square and curly brackets, \texttt{\tbs{}abs} for absulute value, \texttt{\tbs{}norm} for norm, \texttt{\tbs{}fact} for factorial and \texttt{\tbs{}diff} for up-right differential.
    % $$
    %     \rbk{\frac{\pi}{2}}, \quad \sbk{\frac{\pi}{2}}, \quad \cbk{\frac{\pi}{2}}, \quad \abs{\frac{\pi}{2}}, \quad \norm{\frac{\pi}{2}}, \quad \fact{n} = \prod_{i = 1}^{n} i, \quad \frac{\diff \bm{x}}{\diff t} = \bm{v}
    % $$

    % Here are some examples showcasing what is possible with the default packages of \texttt{sleek}.

    % \begin{equation}\label{eq:gauss_law}
    %     \oiint_S \bm{E} \cdot \diff \bm{s} = \iiint_V \frac{\rho}{\varepsilon_0} \diff V
    % \end{equation}

    % \begin{equation*}
    %     e = \sum_{n=0}^\infty \frac{1}{n!}
    % \end{equation*}

    % \begin{subequations}
    %     \begin{align}
    %         \frac{\diff x}{\diff t} & = \alpha x - \beta xy \\
    %         \frac{\diff y}{\diff t} & = \delta xy - \gamma y
    %     \end{align}
    % \end{subequations}

    % \begin{align*}
    %     \ln \abs{x} + C & = \int \frac{1}{x} \,\diff x \\
    %     \exp(x) & = \lim_{n \to \infty} \rbk{1 + \frac{x}{n}}^n
    % \end{align*}

    % \begin{equation}
    %     \left\{
    %     \begin{aligned}
    %         x & = r \sin \theta \cos \phi \\
    %         y & = r \sin \theta \sin \phi \\
    %         z & = r \cos \theta
    %     \end{aligned}
    %     \right.
    % \end{equation}

    % \begin{alignat*}{2}
    %                           & & P(A, B)  & = P(A \mid B) P(B)                        \\
    %     \Leftrightarrow \quad & & P(A \mid B) & = \frac{P(A, B)}{P(B)}                 \\
    %                           & &          & = P(B \mid A) \frac{P(A)}{P(B)}
    % \end{alignat*}

    % \subsection{Units}

    % The \texttt{siunitx} package provides three commands to typeset numbers and quantities -- \texttt{\tbs{}num}, \texttt{\tbs{}si} and \texttt{\tbs{}SI} -- as well as various units (\emph{cf.} Table \ref{tab:siunitx_units}).

    % It is possible to write, both in text or math modes, numbers without units (\emph{e.g.} \num{1}, \num{1.0}, \num{-1}, \num{3.14159}, \num{e100}, $N_A = \num{6.022e23}$), units without quantity (\emph{e.g.} $\si{\joule} = \si{\newton\meter} = \si{\kilogram\meter\squared\per\second\squared}$) and, finally, quantities with their units (\emph{e.g.} \SI{9.81}{\meter\per\second\squared}, $c = \SI{299.6e6}{\meter\per\second}$).

    % \subsection{Lists}

    % Sleek Template uses \texttt{enumitem} to enhance the listing capabilities of \LaTeX{}. There are three lists environments :
    % \begin{enumerate}
    %     \item \texttt{itemize} for unordered lists;
    %     \item \texttt{enumerate} for ordered lists;
    %     \item \texttt{description} for descriptive lists.
    % \end{enumerate}

    % In a list, each element is preceded by the command \texttt{\tbs{}item}. It is possible to modify the labels
    % \begin{itemize}
    %     \item individually with \texttt{\tbs{}item[newLabel]};
    %     \item for the whole environment with the \texttt{label=newLabel} option.
    % \end{itemize}

    % In the case of \texttt{enumerate}, \texttt{newLabel} can contain special expressions (\emph{cf.} Table \ref{tab:enumerate_special_expressions}) that will adapt to the item number. For example, \texttt{label=(\tbs{}alph*)} defines the label sequence \enquote{(a), (b), (c), ...}. Still in the case of \texttt{enumerate}, the  \texttt{\tbs{}setcounter} and \texttt{\tbs{}addtocounter} commands allow to modify the current item number.

    % One could want to reduce the space between items with the \texttt{noitemsep} option or to delete the left margin with the \texttt{leftmargin=*} option.

    % It is also possible to write nested lists. Here follows a very condensed example.

    % \begin{itemize}[leftmargin=*]
    %     \item Lorem ipsum dolor sit amet, consectetur adipiscing elit, sed do eiusmod tempor incididunt ut labore et dolore magna aliqua.

    %     Arcu ac tortor dignissim convallis aenean et tortor. In eu mi bibendum neque egestas congue quisque.

    %     \item[$+$] Semper quis lectus nulla at volutpat diam ut. Felis eget velit aliquet sagittis id. Blandit aliquam etiam erat velit scelerisque in dictum non consectetur.
    %     \begin{equation}
    %         a^2 + b^2 = c^2
    %     \end{equation}

    %     \item Nibh sed pulvinar proin gravida hendrerit lectus. Pretium aenean pharetra magna ac placerat vestibulum lectus mauris. Non consectetur a erat nam at lectus urna duis.
    %     \begin{enumerate}[noitemsep, label=\roman*.]
    %         \item Nibh tortor id aliquet lectus. Sit amet justo donec enim diam vulputate ut pharetra sit.
    %         \setcounter{enumi}{3}
    %         \item Condimentum id venenatis a condimentum vitae. Quis eleifend quam adipiscing vitae proin sagittis nisl.
    %         \addtocounter{enumi}{15}
    %         \item Proin sagittis nisl rhoncus mattis rhoncus urna neque viverra.
    %     \end{enumerate}

    %     \item Elit scelerisque mauris pellentesque pulvinar pellentesque habitant morbi tristique senectus.
    %         \begin{description}
    %             \item[Ridiculus] mus mauris vitae ultricies leo. Mollis aliquam ut porttitor leo a diam. Velit egestas dui id ornare arcu odio ut sem nulla.
    %             \item[Nullam vehicula] ipsum a arcu. Nibh sit amet commodo nulla facilisi nullam. At erat pellentesque adipiscing commodo elit. Libero volutpat sed cras ornare arcu dui.
    %         \end{description}
    % \end{itemize}

    % \subsection{Figures}

    % Thanks to the \texttt{graphicx} package, it is possible to include external graphic documents (images, plots, etc.) in your document with the \texttt{\tbs{}includegraphics} command. Most image type format (\texttt{jpg}, \texttt{png}, \texttt{bmp}, etc.) are supported by this command. However, it should be noted that it is highly preferable to use vectorial types, such as \texttt{pdf} or \texttt{eps}.

    % \begin{figure}[H]
    %     \centering
    %     \includegraphics[width=0.5\textwidth]{resources/pdf/logo.pdf}
    %     \noskipcaption{Random University logo.}
    %     \label{fig:random_university_logo}
    % \end{figure}

    % \subsection{Tables}

    % The packages \texttt{multicol} and \texttt{multirow} comes handy for complex table formatting such as multi-column or multi-row cells.

    % \begin{table}[H]
    %     \centering
    %     \begin{tabular}{|r|r|c|l|}
    %         \hline
    %         \multicolumn{3}{|l|}{a} & qrs  \\ \hline
    %          b &  ef &     jkl      & tuvx \\ \hline
    %         cd & ghi &     mnop     & wyz  \\ \hline
    %     \end{tabular}
    %     \caption{Example of multi-column cells.}
    %     \label{tab:multicol_example}
    % \end{table}

    % \begin{table}[H]
    %     \centering
    %     \begin{tabular}{|l|c|r|}
    %         \hline
    %         \multirow{3}{2cm}{a} &   b   &    c \\ \cline{2-3}
    %                              &  de   &   fg \\ \cline{2-3}
    %                              &  hij  &  klm \\ \hline
    %         nopq                 & rstuv & wxyz \\ \hline
    %     \end{tabular}
    %     \caption{Example of multi-row cells.}
    %     \label{tab:multirow_example}
    % \end{table}

    % \newpage

    % \section{\texttt{sleek-title}}

    % Sleek Template offers a custom title-page with the package \texttt{sleek-title}. The formatting of the title-page is automatically inferred from the fields that the user has provided.

    % The fields are \texttt{\tbs{}logo}, \texttt{\tbs{}institute}, \texttt{\tbs{}faculty}, \texttt{\tbs{}department}, \texttt{\tbs{}title}, \texttt{\tbs{}subtitle}, \texttt{\tbs{}author}, \texttt{\tbs{}supervisor}, \texttt{\tbs{}context} and \texttt{\tbs{}date}.

    % Among these, only \texttt{\tbs{}title}, \texttt{\tbs{}author} and \texttt{\tbs{}date} have to be provided. However, none of the fields should stay empty. Prefer deleting or commenting the line if so.

    % \begin{lstlisting}[style=latexFrameTB, caption={Example of \texttt{sleek-title} title-page definition.}, gobble=8]
    %     \logo{./resources/pdf/logo.pdf}
    %     \institute{Random University}
    %     \faculty{Faculty of Whatever Sciences}
    %     %\department{Department of Anything but Psychology}
    %     \title{A sleek \LaTeX{} template}
    %     \subtitle{With a sleeker title-page}
    %     \author{\textit{Author}\\Francois \textsc{Rozet}}
    %     %\supervisor{Linus \textsc{Torvalds}}
    %     %\context{Well, I was bored...}
    %     \date{\today}
    % \end{lstlisting}

    % It is also possible to use Sleek Template without \texttt{sleek-title}, in which case the default \LaTeX{} title-page will be used.

    % \newpage

    % \section{\texttt{sleek-theorems}}

    % \texttt{sleek-theorems} is based on the \texttt{amsthm} and \texttt{thmtools} packages. It provides a handful of theorem-like environments, each of which have different style and purpose.

    % The environments are \texttt{thm} (theorem), \texttt{lem} (lemma), \texttt{prop} (proposition), \texttt{proof}, \texttt{defn} (definition), \texttt{hyp} (hypothesis), \texttt{meth} (method), \texttt{quest} (question), \texttt{answ} (answer), \texttt{expl} (example), \texttt{rmk} (remark), \texttt{note} and \texttt{tip}.

    % \begin{note}
    %     The option \texttt{french} translates the name of each provided environment. It is also possible, and easy, to add your own language as an option in the source code.
    % \end{note}

    % \begin{thm}[Triangle inequality]
    %     Let be a triangle in Euclidean space. Then the sum of the lengths of two of its sides always surpass or equals the length of the third.
    % \end{thm}

    % \begin{proof}
    %     Let $a$, $b$ and $c$ be the lengths of the sides of a triangle in Euclidean space and $\alpha$, $\beta$, $\gamma$ their respective opposite angle. By the generalized Pythagoras' theorem, we have
    %     \begin{alignat*}{2}
    %                               &  & c^2 & = a^2 + b^2 - 2ab \cos\gamma \\
    %                               &  &     & \leq a^2 + b^2 + 2ab         \\
    %                               &  &     & \leq (a + b)^2               \\
    %         \Leftrightarrow \quad &  & c   & \leq a + b
    %     \end{alignat*}
    %     Therefore in any triangle, the sum of the lengths of two sides always surpass or equals the length of the third.
    % \end{proof}

    % In addition, these environments also have framed versions -- \texttt{framedthm}, \texttt{framedlem}, etc. -- for readability.

    % \begin{framedthm}[Triangle inequality]\label{thm:Triangle inequality}
    %     Let be a triangle in Euclidean space. Then the sum of the lengths of two of its sides always surpass or equals the length of the third.
    % \end{framedthm}

    % \begin{framedprf}
    %     Let $a$, $b$ and $c$ be the lengths of the sides of a triangle in Euclidean space and $\alpha$, $\beta$, $\gamma$ their respective opposite angle. By the generalized Pythagoras' theorem, we have
    %     \begin{alignat*}{2}
    %                               &  & c^2 & = a^2 + b^2 - 2ab \cos\gamma \\
    %                               &  &     & \leq a^2 + b^2 + 2ab         \\
    %                               &  &     & \leq (a + b)^2               \\
    %         \Leftrightarrow \quad &  & c   & \leq a + b
    %     \end{alignat*}
    %     Therefore in any triangle, the sum of the lengths of two sides always surpass or equals the length of the third. \qedadd
    % \end{framedprf}

    % \begin{framedquest*}
    %     Based on the theorem \ref{thm:Triangle inequality}, what is the shortest path from a point $A$ to a point $B$ in Euclidean geometry ?
    % \end{framedquest*}

    % \newpage

    % \section{\texttt{sleek-listings}}

    % The \texttt{sleek-listings} package is a small collection of styles for the environments of the \texttt{listings} package, which is useful to showcase nicely samples of code.

    % Currently, the basic styles \texttt{default} and \texttt{monokai} are implemented. The former is a neutral style (no line numbering, no frame, only good old black code) while the latter is an all-framed style reproducing the iconic \href{https://monokai.nl/}{Monokai} color-map.

    % In addition, the language styles \texttt{c}, \texttt{cpp}, \texttt{matlab}, \texttt{python} and \texttt{java} are implemented, with basic color-maps.

    % Finally, some commands to build upon existing styles are provided :

    % \begin{itemize}
    %     \item \texttt{\tbs{}NumberStyle\{stylename\}} creates a style \texttt{stylenameNumber} with line numbering;
    %     \item \texttt{\tbs{}FrameStyle\{stylename\}} creates a style \texttt{stylenameFrame} with an all around frame;
    %     \item \texttt{\tbs{}FrameTBStyle\{stylename\}} creates a style \texttt{stylenameFrameTB} with top and bottom line rules;
    %     \item \texttt{\tbs{}FrameNumberStyle\{stylename\}} and \texttt{\tbs{}FrameTBNumberStyle\{stylename\}} have the same logic.
    % \end{itemize}

    % For example, the \texttt{\tbs{}FrameTBStyle\{python\}} command creates the \texttt{pythonFrameTB} style, which can then be used to showcase \texttt{Python} code.

    % \FrameTBStyle{python}
    % \begin{lstlisting}[style=pythonFrameTB, gobble=4]
    % import numpy as np # Unnecessary import

    % a, b = 69., .420

    % def f(a: float, b: float) -> float:
    %     r"""
    %     Sum two numbers

    %     Parameters
    %     ----------
    %     a: first number
    %     b: second number

    %     Returns
    %     -------
    %     the sum of 'a' and 'b'
    %     """

    %     return a + b

    % c = f(a, b)

    % print('{:f} + {:f} equals {:f}'.format(a, b, c))
    % \end{lstlisting}

    % \printbibliography

    % \appendix

    % \chapter{Tables}

    % \begin{table}[h]
    %     \centering
    %     \begin{tabular}{ll}
    %         \toprule
    %         \textbf{Package} & \textbf{Purpose} \\
    %         \midrule
    %         \texttt{amsmath} & Mathematical typesetting \\
    %         \texttt{amsthm} & Mathematical environments for theorems, proofs, etc. \\
    %         \texttt{booktabs} & Weighted rules for tables \\
    %         \texttt{biblatex} & Bibliography \\
    %         \texttt{csquotes} & Inline and display quotations \\
    %         \texttt{enumitem} & Lists and enumerations \\
    %         \texttt{float} & Floating objects such as figures and tables \\
    %         \texttt{graphicx} & Graphics \\
    %         \texttt{hyperref} & Hyperlinks and bookmarks \\
    %         \texttt{listings} & Code listings \\
    %         \texttt{multicol} & Table cells that span multiple columns \\
    %         \texttt{multirow} & Table cells that span multiple rows \\
    %         \texttt{siunitx} & Typesetting of units  \\
    %         \texttt{subcaption} & Sub-figures and sub-captions \\
    %         \bottomrule
    %     \end{tabular}
    %     \caption{List of the most relevant packages imported by Sleek Template.}
    %     \label{tab:sleek_relevant_packages}
    % \end{table}

    % \begin{table}[h]
    %     \centering
    %     \begin{tabular}{>{\ttfamily\tbs{}}ll}
    %         \toprule
    %         emph\{abcABC123\} & \emph{abcABC123} \\
    %         bfseries\{abcABC123\} & \bfseries{abcABC123} \\
    %         itshape\{abcABC123\} & \itshape{abcABC123} \\
    %         lowercase\{abcABC123\} & \lowercase{abcABC123} \\
    %         normalfont\{abcABC123\} & \normalfont{abcABC123} \\
    %         textbf\{abcABC123\} & \textbf{abcABC123} \\
    %         textit\{abcABC123\} & \textit{abcABC123} \\
    %         textsc\{abcABC123\} & \textsc{abcABC123} \\
    %         textsf\{abcABC123\} & \textsf{abcABC123} \\
    %         textsl\{abcABC123\} & \textsl{abcABC123} \\
    %         textsubscript\{abcABC123\} & \textsubscript{abcABC123} \\
    %         textsuperscript\{abcABC123\} & \textsuperscript{abcABC123} \\
    %         texttt\{abcABC123\} & \texttt{abcABC123} \\
    %         underline\{abcABC123\} & \underline{abcABC123} \\
    %         uppercase\{abcABC123\} & \uppercase{abcABC123} \\
    %         \bottomrule
    %     \end{tabular}
    %     \caption{Available text fonts in \LaTeX{}.}
    %     \label{tab:text_fonts}
    % \end{table}

    % \begin{table}[h]
    %     \centering
    %     \begin{tabular}{>{\ttfamily\$\tbs{}}l<{\$}l}
    %         \toprule
    %         mathcal\{abcABC123\} & $\mathcal{abcABC123}$ \\
    %         mathit\{abcABC123\} & $\mathit{abcABC123}$ \\
    %         mathnormal\{abcABC123\} & $\mathnormal{abcABC123}$ \\
    %         mathrm\{abcABC123\} & $\mathrm{abcABC123}$ \\
    %         mathbb\{abcABC123\} & $\mathbb{abcABC123}$ \\
    %         mathfrak\{abcABC123\} & $\mathfrak{abcABC123}$ \\
    %         \bottomrule
    %     \end{tabular}
    %     \caption{Available math fonts in \LaTeX{} and AMS.}
    %     \label{tab:math_fonts}
    % \end{table}

    % \begin{table}[h]
    %     \centering
    %     \begin{tabular}{>{\ttfamily}lr|>{\ttfamily}lr|>{\ttfamily}lr}
    %         \toprule
    %         metre & \si{\metre} & second & \si{\second} & mole & \si{\mole} \\
    %         meter & \si{\meter} & ampere & \si{\ampere} & candela & \si{\candela} \\
    %         kilogram & \si{\kilogram} & kelvin & \si{\kelvin} &  &  \\
    %         \midrule
    %         hertz & \si{\hertz} & farad & \si{\farad} & lumen & \si{\lumen} \\
    %         newton & \si{\newton} & ohm & \si{\ohm} & lux & \si{\lux} \\
    %         pascal & \si{\pascal} & siemens & \si{\siemens} & becquerel & \si{\becquerel} \\
    %         joule & \si{\joule} & weber & \si{\weber} & gray & \si{\gray} \\
    %         watt & \si{\watt} & tesla & \si{\tesla} & sievert & \si{\sievert} \\
    %         coulomb & \si{\coulomb} & henry & \si{\henry} &  &  \\
    %         volt & \si{\volt} & celsius & \si{\celsius} &  &  \\
    %         \midrule
    %         angstrom & \si{\angstrom} & day & \si{\day} & liter & \si{\liter} \\
    %         arcminute & \si{\arcminute} & degree & \si{\degree} & litre & \si{\litre} \\
    %         arcsecond & \si{\arcsecond} & electronvolt & \si{\electronvolt} & minute & \si{\minute} \\
    %         barn & \si{\barn} & gram & \si{\gram} & neper & \si{\neper} \\
    %         bar & \si{\bar} & hectare & \si{\hectare} & tonne & \si{\tonne} \\
    %         bel & \si{\bel} & hour & \si{\hour} &  &  \\
    %         \midrule
    %         yocto & \si{\yocto} & milli & \si{\milli} & mega & \si{\mega} \\
    %         zepto & \si{\zepto} & centi & \si{\centi} & giga & \si{\giga} \\
    %         atto & \si{\atto} & deci & \si{\deci} & tera & \si{\tera} \\
    %         femto & \si{\femto} & deca & \si{\deca} & peta & \si{\peta} \\
    %         pico & \si{\pico} & deka & \si{\deka} & exa & \si{\exa} \\
    %         nano & \si{\nano} & hecto & \si{\hecto} & zetta & \si{\zetta} \\
    %         micro & \si{\micro} & kilo & \si{\kilo} & yotta & \si{\yotta} \\
    %         \bottomrule
    %     \end{tabular}
    %     \caption{Available units in the \texttt{siunitx} package.}
    %     \label{tab:siunitx_units}
    % \end{table}

    % \begin{table}[h]
    %     \centering
    %     \begin{tabular}{ll}
    %         \toprule
    %         \textbf{Expression} & \textbf{Description} \\
    %         \midrule
    %         \texttt{\tbs{}arabic*} & Arabic numbers (1, 2, 3, ...) \\
    %         \texttt{\tbs{}alph*} & Lowercase letters (a, b, c, ...) \\
    %         \texttt{\tbs{}Alph*} & Uppercase letters (A, B, C, ...) \\
    %         \texttt{\tbs{}roman*} & Lowercase Roman numerals (i, ii, iii, ...) \\
    %         \texttt{\tbs{}Roman*} & Lowercase Roman numerals (I, II, III, ...) \\
    %         \bottomrule
    %     \end{tabular}
    %     \caption{Special expressions for the label of \texttt{enumerate} environments.}
    %     \label{tab:enumerate_special_expressions}
    % \end{table}

\end{document}
